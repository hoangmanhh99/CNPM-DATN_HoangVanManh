%\makeglossaries
\makenoidxglossaries

% Danh mục thuật ngữ và từ viết tắt

\newglossaryentry{API}{
    type=\acronymtype,
    name={API},
    description={Giao diện lập trình ứng dụng (Application Programming Interface)},
    first={API}
}
\newglossaryentry{Browser}{
    type=\acronymtype,
    name={Browser},
    description={Trình duyệt},
    first={Browser}
}
\newglossaryentry{Real-time}{
    type=\acronymtype,
    name={Real-time},
    description={Thời gian thực},
    first={Thời gian thực}
}
\newglossaryentry{Hot reload}{
    type=\acronymtype,
    name={Hot reload},
    description={Tính năng cho phép các nhà phát triển sửa đổi mã nguồn của ứng dụng trong khi ứng dụng đó đang chạy},
    first={Tính năng cho phép các nhà phát triển sửa đổi mã nguồn của ứng dụng trong khi ứng dụng đó đang chạy}
}
\newglossaryentry{Hot restart}{
    type=\acronymtype,
    name={Hot restart},
    description={Tính năng tải lại toàn bộ thay đổi, khởi động lại ứng dụng},
    first={Tính năng tải lại toàn bộ thay đổi, khởi động lại ứng dụng}
}
\newglossaryentry{Framwork}{
    type=\acronymtype,
    name={Framwork},
    description={Các đoạn mã được viết sẵn, cấu thành một bộ khung và các thư viện lập trình được đóng gói},
    first={Các đoạn mã được viết sẵn, cấu thành một bộ khung và các thư viện lập trình được đóng gói}
}
\newglossaryentry{Arduino}{
    type=\acronymtype,
    name={Arduino},
    description={Tảng vi mạch thiết kế mở phần cứng và phần mềm},
    first={Tảng vi mạch thiết kế mở phần cứng và phần mềm}
}
\newglossaryentry{Localhost}{
    type=\acronymtype,
    name={Localhost},
    description={Máy chủ chạy trên máy tính cá nhân},
    first={Máy chủ chạy trên máy tính cá nhân}
}
\newglossaryentry{Non-blocking}{
    type=\acronymtype,
    name={Non-blocking},
    description={Trong mô hình, các dòng lệnh không nhất thiết phải thực hiện một cách tuần tự và đồng bộ với nhau},
    first={Trong mô hình, các dòng lệnh không nhất thiết phải thực hiện một cách tuần tự và đồng bộ với nhau}
}
\newglossaryentry{Event-driven}{
    type=\acronymtype,
    name={Event-driven},
    description={Kiến trúc hướng sự kiện},
    first={Kiến trúc hướng sự kiện}
}
\newglossaryentry{Codebase}{
    type=\acronymtype,
    name={Codebase},
    description={Phần nội dung hoàn chỉnh của mã nguồn},
    first={Phần nội dung hoàn chỉnh của mã nguồn}
}
\newglossaryentry{Server}{
    type=\acronymtype,
    name={Server},
    description={Một máy được kết nối với mạng máy tính hoặc Internet, có IP tĩnh và năng lực xử lý cao},
    first={Một máy được kết nối với mạng máy tính hoặc Internet, có IP tĩnh và năng lực xử lý cao}
}
\newglossaryentry{Mobile}{
    type=\acronymtype,
    name={Mobile},
    description={Điện thoại di động},
    first={Điện thoại di động}
}
\newglossaryentry{Frontend}{
    type=\acronymtype,
    name={Frontend},
    description={Phần tương tác với người dùng},
    first={Phần tương tác với người dùng}
}
\newglossaryentry{Backend}{
    type=\acronymtype,
    name={Backend},
    description={Những phần bao gồm: máy chủ, ứng dụng và cơ sở dữ liệu},
    first={Những phần bao gồm: máy chủ, ứng dụng và cơ sở dữ liệu}
}
\newglossaryentry{Android}{
    type=\acronymtype,
    name={Android},
    description={Hệ điều hành},
    first={Hệ điều hành}
}
\newglossaryentry{iOS}{
    type=\acronymtype,
    name={iOS},
    description={Hệ điều hành},
    first={Hệ điều hành}
}