\documentclass[../DoAn.tex]{subfiles}
\begin{document}

\begin{center}
    \Large{\textbf{TÓM TẮT NỘI DUNG ĐỒ ÁN}}\\
\end{center}
\vspace{1cm}

Trong cuộc cách mạng công nghệ 4.0 hiện nay, sự bùng nổ của các ngành công nghệ thông tin, điện tử đã làm cho đời sống của con người ngày càng hoàn thiện. Theo báo cáo của IoT Analytics, trong năm 2022, số lượng các thiết bị IoT được kết nối tăng 18\% lên 14,4 tỷ trên toàn cầu \cite{IoTAnalytics}. Có thể thấy rằng các thiết bị thông minh đã được thiết kế, sản xuất ngày càng nhiều và đã bao phủ toàn cầu. Ngay xung quanh chúng ta, các thiết bị thông minh, các cảm biến, dự báo đang được ứng dụng vào cuộc sống sinh hoạt hàng ngày của mỗi người ngày càng nhiều hơn và đã trở thành các thiết bị thiết yếu phục vụ cuộc sống của họ. Đặc biệt, điện thoại di động đã trở thành một phần quen thuộc trong cuộc sống thường nhật của mỗi cá nhân và nhu cầu ứng dụng các tính năng của điện thoại di động vào đời sống ngày càng thiết thực.

Với nhu cầu thông minh hoá các thiết bị điện tử trong đời sống hằng ngày, thay vì phải thực hiện các thao tác điều khiển trực tiếp trên thiết bị hay điểu khiển thiết bị qua bluetooth trong một phạm vi ngắn, trong ĐATN này, em hướng tới xây dựng một ứng dụng có thể áp dụng trong thực tiễn đó là: sử dụng cấu trúc của một hệ thống IoT giúp điều khiển thiết bị Arduino bằng ứng dụng đa nền tảng (Android, iOS, Web) thông qua mạng Internet. Việc xây dựng hệ thống IoT này là nền tảng cho việc phát triển một hệ thống IoT rộng lớn, bền vững hơn và một phần nào đó giúp nâng cao chất lượng cuộc sống con người và đáp ứng các nhu cầu cuộc sống ngày càng tăng lên mạnh mẽ trong thời đại công nghệ số. 
\end{document}